\documentclass[12pt,letterpaper]{article}
\usepackage{fullpage}
\usepackage[top=2cm, bottom=4.5cm, left=2.5cm, right=2.5cm]{geometry}
\usepackage{amsmath,amsthm,amsfonts,amssymb,amscd}
\usepackage{lastpage}
\usepackage{enumitem}
\usepackage{fancyhdr}
\usepackage{mathrsfs}
\usepackage{xcolor}
\usepackage{graphicx}
\usepackage{listings}
%\usepackage{hyperref}
\usepackage[colorlinks=true]{hyperref}
\usepackage{bookmark}
\usepackage{listing}
%\usepackage{appendix}
\usepackage{xcolor}
\usepackage{siunitx}
\usepackage{microtype}
\usepackage{float}
\usepackage{titling}
\usepackage{svg}
\usepackage{longtable}
\usepackage[title,titletoc]{appendix}
\usepackage[UKenglish]{isodate}
\usepackage{derivative}

\definecolor{codegreen}{rgb}{0,0.6,0}
\definecolor{codegray}{rgb}{0.5,0.5,0.5}
\definecolor{codepurple}{rgb}{0.58,0,0.82}
\definecolor{backcolour}{rgb}{0.95,0.95,0.92}

\lstdefinestyle{mystyle}{
	backgroundcolor=\color{backcolour},   
	commentstyle=\color{codegreen},
	keywordstyle=\color{magenta},
	numberstyle=\tiny\color{codegray},
	stringstyle=\color{codepurple},
	basicstyle=\ttfamily\footnotesize,
	breakatwhitespace=false,         
	breaklines=true,                 
	captionpos=b,                    
	keepspaces=true,                 
	numbers=left,                    
	numbersep=5pt,                  
	showspaces=false,                
	showstringspaces=false,
	showtabs=false,                  
	tabsize=2
}

\cleanlookdateon% Remove ordinal day reference

\lstset{style=mystyle}
\setlength{\parindent}{0.0in}
\setlength{\parskip}{0.05in}
\pagestyle{fancyplain}
\headheight 35pt
\rhead{\theauthor\\\today}               
\chead{\textbf{\Large Homework \homeworknumber{}}}
\lhead{\coursenumber \\ \coursename{}}
\lfoot{}
\cfoot{}
\rfoot{\small\thepage}
\headsep 1.5em
\renewcommand{\headrulewidth}{2pt}



% Edit these as appropriate
\author{Evan Burke}
\newcommand\homeworknumber{1} 
\title{Homework \homeworknumber}
\newcommand\coursenumber{AEE 556}
\newcommand\coursename{Compressible Flow}
\newcommand\instructor{Dr. Carson Running}
\newcommand\studentID{101318838}         
%\newcommand\Name{\author}                

\begin{document}
	\begin{titlepage}
	\newcommand{\HRule}{\rule{\linewidth}{0.5mm}}
	
	\begin{figure}
		\centering
		\includesvg[scale=0.85]{images/University_of_Dayton}\\[1cm]
	\end{figure}
	
	\center 
	\quad\\[1.5cm]
	\textsl{\Large \coursenumber\ \textemdash\ \coursename}\\[0.5cm] 
	\textsl{\large Department of Mathematics}\\[0.5cm] 
	\makeatletter
	\HRule \\[0.4cm]
	{ \huge \bfseries \@title}\\[0.4cm] 
	\HRule \\[1.5cm]
	\begin{minipage}{0.4\textwidth}
		\begin{flushleft} \large
			\emph{Author:}\\
			\@author 
		\end{flushleft}
	\end{minipage}
	~
	\begin{minipage}{0.4\textwidth}
		\begin{flushright} \large
			\emph{Instructor:} \\
			\textup{\instructor}
		\end{flushright}
	\end{minipage}\\[3cm]
	\makeatother
	%{\large An Assignment submitted for the UoS:}\\[0.5cm]
	%{\large \emph{Place Your Course Code and Course Name Here}}\\[0.5cm]
	\vfill
	{\large \today}\\[2cm]
	\vfill 
\end{titlepage}
	
	\tableofcontents
	
	%    \maketitle
	
	%\thispagestyle{empty}
	\newpage
	
	\section*{Nomenclature}
	
	{\renewcommand\arraystretch{1.0}
		\noindent\begin{longtable}{@{}l @{\quad=\quad} l@{}}
			$A$  & amplitude of oscillation \\
			$a$ &    cylinder diameter \\
			$C_p$& pressure coefficient \\
			$Cx$ & force coefficient in the \textit{x} direction \\
			$Cy$ & force coefficient in the \textit{y} direction \\
			c   & chord \\
			d$t$ & time step \\
			$Fx$ & $X$ component of the resultant pressure force acting on the vehicle \\
			$Fy$ & $Y$ component of the resultant pressure force acting on the vehicle \\
			$f, g$   & generic functions \\
			$h$  & height \\
			$i$  & time index during navigation \\
			$j$  & waypoint index \\
			$K$  & trailing-edge (TE) nondimensional angular deflection rate
	\end{longtable}}
	
	\addcontentsline{toc}{section}{Nomenclature}
	\newpage
	
	\section{Problem 1}
	In an inviscid flow, a small change in pressure d$p$, is related to a small change in velocity, d$u$, by 
	\begin{equation*}
		dp = - \rho udu    
	\end{equation*}
	
	which is referred to as Euler’s equation and is derived from the conservation of momentum.
	%\addcontentsline{toc}{section}{Problem 1}
	\subsection{}
	Using this relation, derive a differential relation for the fractional density change dρ/ρ as
	a function of the fractional change in velocity du/u, with the fluid’s compressibility τ as a
	coefficient.
	\subsection{}
	\odv{c}{x}
	\subsection{}
	\subsection{}
	\subsection{}
	\subsection{}
	\newpage
	
	\section{Problem 2}
	%\addcontentsline{toc}{section}{Problem 2}
	\subsection{Problem Statement}
	\newpage
	
	\section{Problem 3}
	%\addcontentsline{toc}{section}{Problem 3}
	\subsection{Problem Statement}
	\newpage
	
	\begin{appendices}
		\section{Problem 1 Python Code}
		
		\newpage
		\section{Problem 2 Python Code}
		\newpage
		\section{Problem 3 Python Code}
		\newpage
	\end{appendices}
	
	
	%\newpage
	%\appendix
	%\section{Python Code}
	%\lstinputlisting[language=Python]{Module1.py}
	
\end{document}