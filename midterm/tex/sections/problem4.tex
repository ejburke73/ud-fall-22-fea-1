\documentclass[../main.tex]{subfiles}

\begin{document}

\problem{4}

Concerning FEA models wih 3-node triangular and 4-node rectangular elements:

\begin{enumerate}[label=\alph*)]
    \item Using words and/or equations, explain why these elements perform poorly in bending.
    \item Does ``refining the mesh'' (adding more of these elements) improve the performance of models with these elements? Explain.
    \item What other approach can be taken to improve the performance of these models? Explain.
\end{enumerate}

\problempart{a}

\textbf{ADD MORE ADD MORE ADD MORE}
3-node triangular elements are known as linear strain triangles.
These elements cause a model to appear overly stiff in bending, requiring a larger-than-realistic moment to generate appropriate bending deflections.


\problempart{b}
\textbf{ADD MORE ADD MORE ADD MORE}
Refining a mesh made up of 3-node triangles and/or 4-node rectangles can improve the accuracy of the model with respect to stiffness and internal strain.
However, there is a computational cost that goes hand-in-hand with refining the mesh in this way.

\problempart{c}

\textbf{ADD MORE ADD MORE ADD MORE}
Adding nodes in between existing nodes on the triangular and rectangular elements transforms the elements from linear strain elements to \textbf{SOMETHING ELSE}.
For the same number of elements, the accuracy with these new elements is significantly greater.
However, there is still an added computational cost from the extra nodes.
Generally, this cost is small compared to the cost of refinement using solely 3 and 4-node elements.

\end{document}